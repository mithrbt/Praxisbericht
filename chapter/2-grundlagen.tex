\chapter{Hintergrund und Ziele des Projekts}
\label{chap:grundlagen}
Das vorliegende Projekt beschäftigt sich mit der Anpassung und Erweiterung eines bestehenden Tools, um es den spezifischen Anforderungen der IAV GmbH gerecht zu machen. Zur Verbesserung interner Prozesse und Abläufe in der Messdatenverwaltung möchte IAV die Idee des für BMW entwickelten RequestChannels adaptieren und mit den gesammelten Erfahrungen für die eigenen Anforderungen umsetzen.
\newline
Um dieses Ziel zu erreichen, müssen spezifische Anforderungen im Hinblick auf das Design, die Funktionalitäten und die nahtlose Integration in die bestehende Infrastruktur berücksichtigt werden. 
\newline
Die folgenden Abschnitte erläutern detailliert die Ziele des Projekts sowie die strategischen und technischen Hintergründe, die dieses Vorhaben leiten.
\section{Hintergrund des Projekts}
Der Hintergrund dieses Projekts liegt in der wachsenden Bedeutung der Digitalisierung. In den letzten Jahren sind zahlreiche Herausforderungen in Bezug auf den Zugang und die Nutzung von Daten aufgetreten. Insbesondere der EU Data Act, der am 11. Januar 2024 in Kraft trat, schafft einen neuen Rechtsrahmen, der den fairen und transparenten Zugang zu Daten sowie deren Nutzung regeln soll. Dies soll vor allem dazu beitragen, die Dominanz großer, globaler Tech-Konzerne im Bereich Datenspeicherung und -verarbeitung zu verringern und einen ausgewogeneren Wettbewerb zu ermöglichen .\cite{EUData_JohnerInstitut,bmdv2024}
\newline
Mittels des EU Data Acts soll ein fairer Wettbewerb geschaffen werden, da aktuell die Dominanz der US-Techkonzerne in Bezug auf Datenspeicherund und -verarbeitung diesen gefährden. \cite{EUData_JohnerInstitut}
\newline
Auch IAV ist direkt vom EU Data Act betroffen und benötigt ein spezifisches Tool, um externe Anfragen nach Daten regelkonform bearbeiten zu können. Dies betrifft insbesondere die gesetzlich vorgeschriebenen Anforderungen an den Datenaustausch und die Transparenz der Nutzung, wie sie im EU Data Act festgelegt sind.
\newline
Neben dem externen Bedarf durch den EU Data Act ist der IAV Merida Request jedoch auch von entscheidender Bedeutung für interne Abläufe bei IAV. Momentan fehlt eine zentrale Plattform, über die Kunden von IAV Anfragen nach Messdaten oder Analysen stellen können, um auf die im Unternehmen gesammelten Daten zuzugreifen. Ein solches System ist unerlässlich, um intern effizienter arbeiten zu können und den Datenfluss zu verbessern. 
\newline
\newline
Im Rahmen dieses Studentenprojekts wird eine Whitelabel-Version des BMW RequestChannel entwickelt, der bereits für BMW existiert. Diese Plattform dient BMW als zentrale Anlaufstelle für Analyseanfragen und könnte in ähnlicher Weise für IAV als Datenmanagement- und Anfragetool eingesetzt werden. Die Whitelabel-Version soll als flexible und anpassbare Lösung entwickelt werden, die sowohl interne als auch externe Anforderungen erfüllt.
\newline
Der BMW RequestChannel ist Teil des BMW \ac{CPIC} und ermöglicht es, neue Analyseanfragen zu verschiedenen Datenquellen zu stellen, wenn keine bestehenden Auswertungen über die \ac{CPIC}-Kanäle verfügbar sind. Auf ähnliche Weise soll der IAV Merida Request nicht nur intern genutzt werden, sondern später auch den Anforderungen des EU Data Act gerecht werden, indem es Anfragen von Dritten entsprechend verwaltet und bearbeitet. 
\newline
Die verschiedenen Use Cases, die mögliche Einsatzszenarien des Tools beschreiben, wurden ausgearbeitet und werden im späteren Verlauf des Berichts näher erläutert. Im Rahmen dieses Studentenprojekts liegt der Schwerpunkt jedoch auf dem ersten Use Case, „Projektinterne Anfragen über den IAV Merida Projektexplorer“. Dieser wird detailliert beschrieben und dient als Grundlage für die technische Implementierung.
\section{Ziele des Projekts}
Das Hauptziel des Projekts besteht darin, das bestehende BMW RequestChannel-Tool durch Whitelabeling an die spezifischen Anforderungen und das Branding der IAV GmbH anzupassen. Whitelabeling bedeutet in diesem Kontext, das Tool so zu modifizieren, dass es optisch, funktional und markentechnisch in die IAV-Umgebung integriert wird. Dies umfasst Änderungen am Benutzerinterface, um es mit den \ac{CI/CD} von IAV anzupassen. Neben der visuellen Anpassung ist es auch notwendig, sicherzustellen, dass die Software den spezifischen technischen und funktionalen Bedürfnissen von IAV entspricht. Dies könnte etwa spezielle Workflows oder Zugriffsrechte beinhalten, die für IAV-Nutzer relevant sind. Das Ziel ist es, das Tool so zu gestalten, dass es nicht nur optisch, sondern auch funktional perfekt in die Arbeitsumgebung und die Prozesskette von IAV integriert wird.
\newline
\newline
Ein weiteres zentrales Ziel des Projekts ist die reibungslose Integration des angepassten RequestChannel-Tools in die bestehende IAV Merida Toolkette. Die IAV Merida Toolkette bietet eine Infrastruktur für die Nutzung und Verwaltung von Messdaten sowie für Analysen und andere Entwicklungsprozesse. Es ist entscheidend, dass das Tool nahtlos in diese Umgebung eingebettet wird, damit Benutzer direkt auf das Tool zugreifen und es effizient in ihre bestehenden Arbeitsprozesse integrieren können. Dies umfasst die Integration der Kommunikationsschnittstellen, die Datenübertragung zwischen den verschiedenen Tools sowie die Sicherstellung, dass das RequestChannel-Tool innerhalb der Merida-Plattform einwandfrei funktioniert. Durch diese Integration wird der Prozess der Messdatenverwaltung und -analyse optimiert.
\newline
\newline
Zudem müssen die vorher definierten Use Cases, die eine Erweiterung der Nutzungsmöglichkeiten des Tools beschreiben, implementiert werden. Diese Use Cases sind oft auf spezielle Arbeitsprozesse, Bedürfnisse oder Herausforderungen innerhalb von IAV zugeschnitten. Die Erweiterung der Funktionalitäten kann beispielsweise die Implementierung zusätzlicher Automatisierungsfunktionen, Berichts- und Analyseoptionen oder die Integration neuer Schnittstellen betreffen. 
\newline
Im Rahmen des Projekts liegt der Fokus auf der Umsetzung des ersten Use Cases. Ziel ist es, eine zentrale Plattform zu schaffen, die Kundenanfragen über den IAV Merida Projektexplorer effizient verwaltet und die Bearbeitung dieser Anfragen verbessert. Weitere Use Cases werden im Hinblick auf eine potenzielle zukünftige Erweiterung betrachtet, jedoch nicht im aktuellen Projekt implementiert.
\newline
Das Ziel ist es, das Tool so zu optimieren, dass es für die IAV-Anwender nicht nur die vorhandenen Anforderungen erfüllt, sondern auch zusätzliche Mehrwerte bietet. Dadurch wird das Tool zu einer flexiblen Lösung, die in verschiedenen Anwendungsbereichen effizient eingesetzt werden kann. Es soll die Möglichkeit bieten, unterschiedliche Arbeitsprozesse zu unterstützen und nahtlos in verschiedene Szenarien integriert zu werden, um den Mehrwert für IAV-Anwender in vielfältigen Einsatzgebieten zu maximieren.
\label{chap:kapitel2}

