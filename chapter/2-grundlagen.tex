\chapter{Hintergrund und Ziele des Projekts}
\label{chap:grundlagen}
Das vorliegende Projekt beschäftigt sich mit der Anpassung und Erweiterung eines bestehenden Tools, um es den spezifischen Anforderungen der IAV GmbH gerecht zu machen. Im Rahmen der Weiterentwicklung interner Prozesse und der Optimierung von Arbeitsabläufen im Bereich der Messdatenverwaltung hat sich die IAV dazu entschieden, das von BMW entwickelte RequestChannel-Tool für eigene Zwecke anzupassen und in die bestehende IAV Merida Toolkette zu integrieren. Diese Anpassung, auch als Whitelabeling bezeichnet, erlaubt es, das Tool optisch und funktional so zu gestalten, dass es als eine inhouse-Lösung erscheint, die perfekt auf die Bedürfnisse der IAV-Nutzer abgestimmt ist.

Um dieses Ziel zu erreichen, müssen spezifische Anforderungen im Hinblick auf das Design, die Funktionalitäten und die nahtlose Integration in die bestehende Infrastruktur berücksichtigt werden. Darüber hinaus ist es ein wesentlicher Bestandteil des Projekts, zu prüfen, ob das derzeit genutzte Ticketingsystem Jira als Backend für das Tool weiterhin geeignet ist, oder ob eine alternative Lösung angestrebt werden sollte. Die folgenden Abschnitte erläutern detailliert die Ziele des Projekts sowie die strategischen und technischen Hintergründe, die dieses Vorhaben leiten.
\section{Hintergrund des Projekts}
Der Hintergrund dieses Projekts liegt in der wachsenden Bedeutung der Digitalisierung. In den letzten Jahren sind zahlreiche Herausforderungen in Bezug auf den Zugang und die Nutzung von Daten aufgetreten. Insbesondere der EU Data Act, der am 11. Januar 2024 in Kraft trat, schafft einen neuen Rechtsrahmen, der den fairen und transparenten Zugang zu Daten sowie deren Nutzung regeln soll. Dies soll vor allem dazu beitragen, die Dominanz großer, globaler Tech-Konzerne im Bereich Datenspeicherung und -verarbeitung zu verringern und einen ausgewogeneren Wettbewerb zu ermöglichen .\cite{EUData_JohnerInstitut,bmdv2024}
\newline
Mittels des EU Data Acts soll zum einen ein fairer Wettbewerb geschaffen werden, da aktuell die Dominanz der US-Techkonzerne in Bezug auf Datenspeicherund und -verarbeitung diesen gefährden. \cite{EUData_JohnerInstitut}
\newline
Auch IAV ist direkt vom EU Data Act betroffen und benötigt ein spezifisches Tool, um externe Anfragen nach Daten regelkonform bearbeiten zu können. Dies betrifft insbesondere die gesetzlich vorgeschriebenen Anforderungen an den Datenaustausch und die Transparenz der Nutzung, wie sie im EU Data Act festgelegt sind.
\newline
Neben dem externen Bedarf durch den EU Data Act ist dieses Tool jedoch auch von entscheidender Bedeutung für interne Abläufe bei IAV. Momentan fehlt eine zentrale Plattform, über die Mitarbeiter von IAV Anfragen nach Messdaten oder Analysen stellen können, um auf die im Unternehmen gesammelten Daten zuzugreifen. Ein solches System ist unerlässlich, um intern effizienter arbeiten zu können und den Datenfluss zwischen verschiedenen Fachbereichen zu verbessern. 
\newline
\newline
Im Rahmen dieses Studentenprojekts wird eine Whitelabel-Version des BMW RequestChannel entwickelt, der bereits für BMW existiert. Diese Plattform dient BMW als zentrale Anlaufstelle für Analyseanfragen und könnte in ähnlicher Weise für IAV als Datenmanagement- und Anfragetool eingesetzt werden. Die Whitelabel-Version soll als flexible und anpassbare Lösung entwickelt werden, die sowohl interne als auch externe Anforderungen erfüllt.
\newline
Der BMW RequestChannel ist Teil des BMW CPIC (Customer and Product Intelligence Center) und ermöglicht es, neue Analyseanfragen zu verschiedenen Datenquellen zu stellen, wenn keine bestehenden Auswertungen über die CPIC-Kanäle verfügbar sind. Auf ähnliche Weise soll das IAV-Tool nicht nur intern genutzt werden, sondern später auch den Anforderungen des EU Data Act gerecht werden, indem es Anfragen von Dritten entsprechend verwaltet und bearbeitet.
\section{Ziele des Projekts}
Das Hauptziel des Projekts besteht darin, das bestehende BMW RequestChannel-Tool durch Whitelabeling an die spezifischen Anforderungen und das Branding der IAV GmbH anzupassen. Whitelabeling bedeutet in diesem Kontext, das Tool so zu modifizieren, dass es optisch, funktional und markentechnisch in die IAV-Umgebung integriert wird. Dies umfasst Änderungen am Benutzerinterface, um es mit den Farben, Logos und Designrichtlinien der IAV zu versehen, sodass es wie ein eigenentwickeltes Tool erscheint. Neben der visuellen Anpassung ist es auch notwendig, sicherzustellen, dass die Software den spezifischen technischen und funktionalen Bedürfnissen der IAV entspricht. Dies könnte etwa spezielle Workflows oder Zugriffsrechte beinhalten, die für IAV-Nutzer relevant sind. Das Ziel ist es, das Tool so zu gestalten, dass es nicht nur optisch, sondern auch funktional perfekt in die Arbeitsumgebung und die Prozesskette der IAV integriert wird.
\newline
\newline
Ein weiteres zentrales Ziel des Projekts ist die reibungslose Integration des angepassten RequestChannel-Tools in die bestehende IAV Merida Toolkette. Die IAV Merida Toolkette bietet eine Infrastruktur für die Nutzung und Verwaltung von Messdaten sowie für Analysen und andere Entwicklungsprozesse. Es ist entscheidend, dass das Tool nahtlos in diese Umgebung eingebettet wird, damit Benutzer direkt auf das Tool zugreifen und es effizient in ihre bestehenden Arbeitsprozesse integrieren können. Dies umfasst die Integration der Kommunikationsschnittstellen, die Datenübertragung zwischen den verschiedenen Tools sowie die Sicherstellung, dass das RequestChannel-Tool innerhalb der Merida-Plattform einwandfrei funktioniert. Durch diese Integration wird der Prozess der Messdatenverwaltung und -analyse optimiert, da Anwender keine externen Tools mehr verwenden müssen, sondern das RequestChannel-Tool als Teil der bereits bekannten und genutzten Infrastruktur nutzen können.
\newline
\newline
Um den spezifischen Anforderungen der IAV-Nutzer gerecht zu werden, ist es notwendig, die Funktionalitäten des Tools zu erweitern. Dies bedeutet, dass die vorher definierten Use Cases, die eine Erweiterung der Nutzungsmöglichkeiten des Tools beschreiben, implementiert werden müssen. Diese Use Cases sind oft auf spezielle Arbeitsprozesse, Bedürfnisse oder Herausforderungen innerhalb der IAV zugeschnitten. Die Erweiterung der Funktionalitäten kann beispielsweise die Implementierung zusätzlicher Automatisierungsfunktionen, Berichts- und Analyseoptionen oder die Integration neuer Schnittstellen betreffen. Das Ziel ist es, das Tool so zu optimieren, dass es für die IAV-Anwender nicht nur die vorhandenen Anforderungen erfüllt, sondern auch zusätzliche Mehrwerte bietet. Damit wird das Tool vielseitiger einsetzbar und kann in verschiedenen Anwendungsbereichen effizient genutzt werden.
\newline
Ein weiteres Ziel des Projekts ist die technische Prüfung, ob Jira als Backend bzw. Ticketingsystem für das RequestChannel-Tool geeignet ist oder ob es sinnvoller wäre, es durch eine Eigenimplementierung oder ein alternatives System zu ersetzen. Jira wird oft als mächtiges und flexibles Ticketing- und Projektmanagement-Tool eingesetzt, jedoch ist zu prüfen, ob es in diesem speziellen Kontext den Anforderungen von IAV gerecht wird. Dies betrifft sowohl die Performance, die Benutzerfreundlichkeit und die Anpassbarkeit an IAV-spezifische Workflows als auch die Kosten und Wartungsanforderungen. Sollte Jira sich als nicht optimal herausstellen, besteht die Option, eine Eigenentwicklung zu realisieren oder nach Alternativen zu suchen, die möglicherweise besser auf die speziellen Bedürfnisse von IAV zugeschnitten sind. Diese Prüfung stellt sicher, dass das eingesetzte Backend-System den optimalen Mehrwert für das Projekt und die gesamte IAV-Infrastruktur bietet.


