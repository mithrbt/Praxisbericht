\chapter{Zusammenfassung und Ausblick}
In diesem Praxisbericht wird die Entwicklung des IAV Merida Request-Tools vorgestellt – einer maßgeschneiderten Lösung zur strukturierten Bearbeitung von Daten- und Analyseanfragen. Ziel des Projekts war es, mithilfe einer Whitelabel-Version des BMW RequestChannels ein Tool zu schaffen, das speziell auf die Anforderungen der IAV GmbH zugeschnitten ist. Die Anwendung ermöglicht es, den Workflow zur Bereitstellung von Daten und Analysen effizienter zu gestalten und so die internen Abläufe zu optimieren. Der Fokus lag dabei auf der Umsetzung des ersten Use Cases „Projektinterne Anfragen über den IAV Merida Projektexplorer“, der es Kunden ermöglicht, gezielt Messdaten und Analysen für spezifische Projekte anzufordern.
\newline
Die Projektarbeit umfasste mehrere Phasen: Zunächst wurden spezifische Use Cases definiert, die als Basis für die Funktionalitäten und das Design des Tools dienten. Anschließend erfolgte die Konzeption der Systemarchitektur und der Benutzeroberfläche, um eine optimale technische Grundlage sowie eine benutzerfreundliche Bedienung zu gewährleisten. Der Designprozess der Mockups spielte eine zentrale Rolle, um das Interface klar und intuitiv zu gestalten und den Workflow für die Endbenutzer zu vereinfachen. Schließlich wurde die Implementierung der Funktionalitäten und des User Interfaces durchgeführt und abgeschlossen, sodass das IAV Merida Request-Tool alle Anforderungen des ersten Use Cases vollständig erfüllt.
\newline
Für zukünftige Projekte bieten sich zahlreiche Erweiterungsmöglichkeiten an, um das Tool weiter auszubauen. So könnte die Integration einer Confluence-Schnittstelle eine automatisierte Dokumentation der Anfragen ermöglichen. Auch der Einsatz eines Ticketsystems wie Jira wäre denkbar, um die Verwaltung und Zuweisung von Anfragen an Analytiker und andere Mitarbeiter effizienter zu gestalten. Diese Erweiterungen könnten die Effizienz und Nachverfolgbarkeit des Anfrageprozesses weiter verbessern und das Tool noch stärker in bestehende Arbeitsabläufe einbinden.